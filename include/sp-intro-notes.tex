
\paragraph*{Notes}
\begin{itemize}
  \setlength\itemsep{0em}
  \item this isn't an exhaustive listing of all features available on Smil, but just the most common features. For a complete listing, please refer to the complete documentation (\IT{RTFM});

  \item some functions may have some other variants on the parameters (number or presence). Most of the time, these functions are marked with a \BF{"*"}. Take a look on the complete documentation;


  \item To make \BF{smilPython} available in your programs you must type one of the following commands:
    \begin{lstlisting}
    from smilPython import *
    import smilPython as sp
    \end{lstlisting}

  \item parameters between square brackets indicates an optional parameter. For example :
    \begin{lstlisting}[language=Python]
    # erode() function prototype : 
    erode(imIn, imOut[, se])
    # function call
    erode(imIn, imOut)
    erode(imIn, imOut, SquSE())
    erode(imIn, imOut, CrossSE(3))
    \end{lstlisting}

  \item if not specified, default Structuring Element is used in most morphological functions, usualy the last one;

  \item \BF{all Smil functions} returns some value : a \BF{data} or a \BF{result} indicating an error condition (\BF{1} means \BF{no error});

  \item output image passed as a parameter \BF{must always} be created (even if not initialized) before any function call.

\end{itemize}
